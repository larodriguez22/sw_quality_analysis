% $Id: results.tex 1784 2012-04-27 23:29:31Z nicolas.cardozo $
% !TEX root = main.tex

\chapter{Analysis and Results}
\label{cha:results}
This section presents the results of the survey presented in \fref{sec:evaluation}. This section 
is divided by the three main sections of the survey. Additionally, a discussion section is added. 

\section{General Knowledge Results}
\label{sec:general-knowledge}
Most of the answers said they were very experienced on the use of python as it
can be seen in \fref{fig:exp-py}, most of the people had a high level of expertise (over 5)
as they used Python very often in their work and university. Most of 
the people were very familiarized using \ac{RL} algorithms (\fref{fig:exp-rl}), as they were 
taking the course. This is very important information, as the bugs require \ac{RL} and python 
knowledge to be identified. Furthermore, the students were very familiarized with 
the terminal, this means it would be easy for them to interact with the tool, as the commands 
and the interface require experience using the console (\fref{fig:exp-terminal}). Nevertheless, 
average the 
students weren't familiarized with debuggers, and they only used them very rarely, or used 
Visual Studio Code interface simple features, like breakpoint function. This lead to 
misunderstandings and probably difficulties in making the tasks, as the tool is based on
previous debugger (\fref{fig:exp-deb}). Also, this made the learning curve of the tool 
steeper than with a person with previous debugger experience.

\section{Tasks Results}
\label{sec:tasks-results}
As there can be seen in the tables \fref{tab:grid-results}, \fref{tab:rooms-results} and 
\fref{tab:cars-results}, most of the people made it through the first task (\fref{fig:task1}),
they thought it was an easy task to solve. This task besides being easy, took them a lot of 
time to complete because this was the task in which they were getting familiarized with the 
tool. The second task was harder than the first one, most of the people took longer to finish the 
task (\fref{fig:task2}), and most of the people had trouble finding the problem to the program. 
This was meant to be as task 2, wasn't meant to be completely wrong, it was just taking really big 
steps, and the error wasn't because a variable definition, but because a changed in the QLearning
algorithm, which made a lot of students initially confused. Finally, the third task (\fref{fig:task3}) was the hardest 
one, most people had trouble finding the bug. They thought it wasn't an easy task to solve, 
nevertheless, most of the people managed to to finish the task in less time than in the other tasks. 
In this case, it is understandable that for the first task besides being 
easy people took longer, because they were getting familiarized with the tool. For the second task,
given the nature of the bug and the fact that the bug was introduced in the QLearning algorithm, 
it was harder to find the bug, and for that people thought it was harder to solve, but
most of the people finish it. Finally, for the third task, the bug was introduced in the rewards,
people took less as they had more practice with the tool.



\section{Debugger Usability Results}
\label{sec:usability}

Regarding the debugger usability (\fref{tab:general1-debuggers} and \fref{tab:general2-debuggers}), it can 
be seen that the tool was useful,
and the general comments were that the tool would be useful if there was more time to study it. 
And to learn about the tool before using it in real tasks. The 
tool was easy to use, once they got to know it a little better, and they got to practice more with the commands.
Additionally, there were several comments about the tool improving the UI,
as it was hard to understand the tool at first sight, it wasn't similar to the Visual Studio Code 
interface that most of the people were used to use.

% \begin{table}[]
% \centering
% \resizebox{\columnwidth}{!}{%
% \begin{tabular}{
% >{\columncolor[HTML]{FFFFFF}}c 
% >{\columncolor[HTML]{FFFFFF}}c 
% >{\columncolor[HTML]{FFFFFF}}c 
% >{\columncolor[HTML]{FFFFFF}}c 
% >{\columncolor[HTML]{FFFFFF}}c 
% >{\columncolor[HTML]{FFFFFF}}c 
% >{\columncolor[HTML]{FFFFFF}}c 
% >{\columncolor[HTML]{FFFFFF}}c 
% >{\columncolor[HTML]{FFFFFF}}c 
% >{\columncolor[HTML]{FFFFFF}}c 
% >{\columncolor[HTML]{FFFFFF}}c }
% \multicolumn{11}{c}{\cellcolor[HTML]{FFFFFF}{\color[HTML]{383838} \textbf{Debugger Usability.}}}                                                                                                                                                                                                                                                                                                                                                                                                                                                                                                                                                                                                                                                                                                                                                                                                                                                                                                                                                                                                                                                                                                                                                                                                                                                         \\
% {\color[HTML]{383838} \textbf{}}  & {\color[HTML]{383838} \textbf{\begin{tabular}[c]{@{}c@{}}I think I would like to use this \\ system frequently\end{tabular}}} & {\color[HTML]{383838} \textbf{\begin{tabular}[c]{@{}c@{}}I find this system \\ unnecessarily complex\end{tabular}}} & {\color[HTML]{383838} \textbf{\begin{tabular}[c]{@{}c@{}}I think the system \\ is easy to use\end{tabular}}} & {\color[HTML]{383838} \textbf{\begin{tabular}[c]{@{}c@{}}I think I would need technical \\ support to use the system\end{tabular}}} & {\color[HTML]{383838} \textbf{\begin{tabular}[c]{@{}c@{}}I find the various functions of the \\ system quite well integrated\end{tabular}}} & {\color[HTML]{383838} \textbf{\begin{tabular}[c]{@{}c@{}}I have found too much \\ inconsistency in this system\end{tabular}}} & {\color[HTML]{383838} \textbf{\begin{tabular}[c]{@{}c@{}}I think most people would learn \\ to use the system quickly\end{tabular}}} & {\color[HTML]{383838} \textbf{\begin{tabular}[c]{@{}c@{}}I found the system quite \\ awkward to use\end{tabular}}} & {\color[HTML]{383838} \textbf{\begin{tabular}[c]{@{}c@{}}I have felt very safe \\ using the system\end{tabular}}} & {\color[HTML]{383838} \textbf{\begin{tabular}[c]{@{}c@{}}I would need to learn a lot \\ of things before I could handle the system\end{tabular}}} \\
% {\color[HTML]{383838} \textbf{5}} & {\color[HTML]{383838} 7}                                                                                                      & {\color[HTML]{383838} 2}                                                                                            & {\color[HTML]{383838} 5.0}                                                                                   & {\color[HTML]{383838} 9}                                                                                                            & {\color[HTML]{383838} 6}                                                                                                                    & {\color[HTML]{383838} 1}                                                                                                      & {\color[HTML]{383838} 7}                                                                                                             & {\color[HTML]{383838} 4}                                                                                           & {\color[HTML]{383838} 7}                                                                                          & {\color[HTML]{383838} 8.0}                                                                                                                        \\
% {\color[HTML]{383838} \textbf{2}} & {\color[HTML]{383838} 7}                                                                                                      & {\color[HTML]{383838} 6}                                                                                            & {\color[HTML]{383838} 6.0}                                                                                   & {\color[HTML]{383838} 3}                                                                                                            & {\color[HTML]{383838} 2}                                                                                                                    & {\color[HTML]{383838} 7}                                                                                                      & {\color[HTML]{383838} 6}                                                                                                             & {\color[HTML]{383838} 4}                                                                                           & {\color[HTML]{383838} 4}                                                                                          & {\color[HTML]{383838} NaN}                                                                                                                        \\
% {\color[HTML]{383838} \textbf{3}} & {\color[HTML]{383838} 7}                                                                                                      & {\color[HTML]{383838} 7}                                                                                            & {\color[HTML]{383838} 11.0}                                                                                  & {\color[HTML]{383838} 4}                                                                                                            & {\color[HTML]{383838} 5}                                                                                                                    & {\color[HTML]{383838} 4}                                                                                                      & {\color[HTML]{383838} 6}                                                                                                             & {\color[HTML]{383838} 9}                                                                                           & {\color[HTML]{383838} 7}                                                                                          & {\color[HTML]{383838} 4.0}                                                                                                                        \\
% {\color[HTML]{383838} \textbf{4}} & {\color[HTML]{383838} 5}                                                                                                      & {\color[HTML]{383838} 7}                                                                                            & {\color[HTML]{383838} 5.0}                                                                                   & {\color[HTML]{383838} 8}                                                                                                            & {\color[HTML]{383838} 13}                                                                                                                   & {\color[HTML]{383838} 2}                                                                                                      & {\color[HTML]{383838} 6}                                                                                                             & {\color[HTML]{383838} 7}                                                                                           & {\color[HTML]{383838} 8}                                                                                          & {\color[HTML]{383838} 6.0}                                                                                                                        \\
% {\color[HTML]{383838} \textbf{1}} & {\color[HTML]{383838} 1}                                                                                                      & {\color[HTML]{383838} 5}                                                                                            & {\color[HTML]{383838} NaN}                                                                                   & {\color[HTML]{383838} 3}                                                                                                            & {\color[HTML]{383838} 1}                                                                                                                    & {\color[HTML]{383838} 13}                                                                                                     & {\color[HTML]{383838} 2}                                                                                                             & {\color[HTML]{383838} 3}                                                                                           & {\color[HTML]{383838} 1}                                                                                          & {\color[HTML]{383838} 9.0}                                                                                                                       
% \end{tabular}%
% }
% \end{table}

\section{Discussion}
\label{sec:discussion}

In general, the results of the survey were very positive, most of the people thought that 
the tool was useful specially for the kind of challenges \ac{RL} programs could have. In 
spite of the tool being hard to familiarized with, specially for the people who had very 
little experience with debuggers, most of the people thought that the tool was easy to use 
once you get to know it a little better. Additionally, people said that the tool would've 
been useful for the course, because of its capabilities to debug \ac{RL} programs.

Finally, in the following links there are two examples of the tool being used. In the first 
example, the tool is used to debug a simple program, this was used to introduced to the students 
the simple commands they could use like stepping (forward and backwards), modifying, or inspecting 
variables. In the second example, the tool is used to debug a \ac{RL} program, specially 
the gridworld example.

\endinput

