% $Id: introduction.tex 1784 2012-04-27 23:29:31Z nicolas.cardozo $
% !TEX root = main.tex

\chapter{State of the Art}
\label{cha:state_of_the_art}

In this chapter, we explore the state of the art in visualization and debugging 
tools for reinforcement learning programs. The most relevant tools are presented, 
and their advantages and disadvantages are discussed.

As a reference solution for the problem outlined in the previous chapter, we 
will consider back-in-time debuggers, which allow the user to go back in time 
and observe the program's state at a previous moment. Therefore, we take 
Delorean \cite{delorean} as a reference, which is a back-in-time debugger for 
JavaScript programs. Most debuggers do not consider this approach of rewinding 
to a specific point in execution, thereby missing out on several opportunities; 
for instance, testing different variable values within the same execution 
context to better understand errors or explore hypothetical scenarios in the 
program's execution evolution. These functionalities are precisely what we are 
interested in working on and understanding in reinforcement learning programs.


Por otro lado, los programas de aprendizaje de maquina (ML) normalmente se debuggean
con herramientas de visualización que permiten a los desarrolladores observar
el comportamiento del programa y analizar su estado interno. Sin embargo, estas
herramientas no son adecuadas para RL, ya que no consideran la naturaleza cíclica
y adaptativa de los programas de RL. Sin embargo, presentan un paso importante del
estado del arte para la creación de herramientas de visualización y debuggeado.

Una de estas herramientas, y sin duda la más relevante, es TensorBoard 
\cite{tensorboard}, una herramienta de visualización para TensorFlow, una de las
bibliotecas de aprendizaje de maquina más populares. TensorBoard permite a los
desarrolladores visualizar el comportamiento de sus modelos de aprendizaje de
maquina, incluyendo gráficos de entrenamiento, distribuciones de variables, y
visualizaciones de la estructura del modelo. Sin embargo, TensorBoard no permite
a los desarrolladores interactuar con el modelo durante la ejecución, lo cual
es esencial para el debuggeado de RL.






\endinput

