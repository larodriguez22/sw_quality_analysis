% $Id: state_of_the_art.tex 1784 2012-04-27 23:29:31Z nicolas.cardozo $
% !TEX root = main.tex

\chapter{State of the Art}
\label{cha:state_of_the_art}

In this chapter, we explore the state of the art, to fully understand the solution 
for the problem aforementioned, this section is structured in four main parts, that 
will cover deep understanding into what a debugger is (\fref{sec:deb}), what is the 
state of the art for debugger in other programming languages (\fref{sec:other}), what 
is the state of the art for python debuggers (\fref{sec:py}), and finally, what is the 
state of the art for debugging \ac{AI} programs, specifically \ac{RL} programs (\fref{sec:ai}).

\section{What is a Debugger?}
\label{sec:deb}
% Definition of bug
% -> Bug types
% Definition of debugging
% -> Principles of debugging
% Definition of Debugger
% -> types of debuggers
% -> main operations
A bug is a wrong behavior in a program that causes it to crash or produce incorrect 
results. These are the type of behavior that we as programmers want to avoid in our
programs. However, bugs are common in software development, and they can be caused by 
many factors, such as incorrect logic, incorrect syntax, or incorrect data. Bugs can
be difficult to find and fix, especially in large and complex programs. For this work 
we will focus specifically in run-time semantic errors, which are the ones that allow 
compilation, but programs still end up in an abnormal termination, and logic 
errors, which allow compilation too, but lead to incorrect results. Debugging is the 
process of finding and reducing the number of 
bugs or defects in a program\cite{10.1007/978-3-642-32729-2_3}, in practical terms, 
debugging is defined as the activity that comes after testing when developers identify 
an error and how to fix it\cite{10.1080/08993400802114581}. 

There are several principles that guide the debugging process, these principles are 
very well explained in the book the Art of debugging with \ac{GDB}, \ac{DDD}, and Eclipse\cite{10.5555/1457534}.
The fundamental principle that drives the essence of debugging is the Principle of 
Confirmation: fixing a program with bugs is a process of confirming that the things you believe
to be true are actually true in your code\cite{10.5555/1457534}, basically, 
when you find that those assumptions are not true, you have found the bug or a clue to it.

A classic debugging technique is adding traces in your code in terms of printing the 
values of several variables as the program is executed, however, this is not the most 
efficient way to debug a program, reason why debuggers exist. A debugger is a tool 
that allows developers to inspect the state of a program while it is running, and to 
control its execution. Debuggers are essential tools for software development, as they 
help developers find and fix bugs in their programs. There are many types of 
debuggers, each with its own set of features and capabilities, but usually the main 
types of operations that a debugger offers\cite{10.5555/1457534} are:

\begin{itemize}
    \item Stepping through code: where you can run the program line by line and step 
    the execution at certain points. This is usually done by using breakpoints, or stepping, 
    and continue functions that allow you to go step by step forward in the execution of
    the program.
    \item Inspecting variables: you can inspect the current values of variables in your 
    program this will usually help you to understand the state of the program and lead you 
    to possible clues on where the bug is.
    \item Changing the program state: you can change the values of variables in your program, 
    this is useful when you want to test different scenarios in your program, note that not
    all debuggers allow this operation.
\end{itemize}

In summary, a debugger is a tool that allows developers to inspect the state of a program,
control its execution, and find and fix bugs. Debuggers are essential tools for software 
development, as they help developers find and fix bugs in their programs. Now as said before, 
we want to understand \ac{RL} programs, in order to create better code and understand the behavior
of the agent; our solution to the problem is to create a debugger for \ac{RL} programs.

\section{State of the art in programming languages}
\label{sec:other}
Now, we will explore the state of the art in debugging tools in general for different 
programming languages. There are many debugging tools available for different programming 
languages, each with its own set of features and capabilities, and most of the IDEs have 
built-in debugging tools that allow you to inspect the state of your program while it is 
running. However, there are also standalone debugging tools that can be used with different 
programming languages, and that offer more advanced features and capabilities than the 
built-in debugging tools of an IDE. In this section we will dive more in depth in different 
debugging tools for different programming languages.

Some of the most popular debugging tools include \ac{GDB}\cite{stallman2011debugging}, 
which stands for GNU Project Debugger and is a powerful 
debugging tool for C and C++ (although it also supports other programming languages,
like Ada, Go, or Rust). GDB helps you to poke around inside C programs while 
they are executing, while it allows you to see what exactly happens when your program 
crashes. \ac{GDB} operates on executable files which are binary files produced by the 
compilation process. \ac{GDB} consists of three major subsystems: first, user interface, which 
consists of several actual interfaces, plus supporting code. Second, symbol handling, 
which consists of object file readers, debugging info interpreters, symbol table 
management, source language expression parsing and type and value printing. Third target 
system handling, consists of execution control, stack frame analysis, and 
physical target manipulation\cite{stallman2011debugging}.

As a reference for the type of debuggers we want to use and create for the solution 
to the problem outlined in the previous chapter, we will consider back-in-time 
debuggers, which allow the user to go back in time and observe the program's state 
at a previous moment. We take DeloreanJs\cite{delorean23} as a reference, 
which is a back-in-time debugger for JavaScript programs. Most debuggers do not 
consider the approach of rewinding to a specific point in execution, thereby 
missing out on several opportunities; for instance, testing different variable 
values within the same execution context to better understand errors or explore 
hypothetical scenarios in the program's execution evolution. These functionalities 
are precisely what we are interested in working on and understanding in 
reinforcement learning programs. DeloreanJs has three main features we are looking 
for in our solution: the ability to navigate through an execution history, modify 
values within a variable, and resume execution from a timepoint in the 
past\cite{delorean23}. To do this, they used an extended version of continuations 
with static analysis, in which the continuations allow to capture the current program 
state, and the static analysis allows to storage variable's values. There are several 
other debuggers for Javascript, but most of them had a breakpoint approach, in which,
they did not let modification of variables and resume of the current execution.

\section{State of the art in Python}
\label{sec:py}
% how is python being excecuted?
Now most of the \ac{RL} programs are usually written in Python. So we searched the state 
of the art of debugging tools for Python. Python is an easy to learn, 
powerful programming language. It has efficient high-level data structures and a simple 
but effective approach to object-oriented programming. Is a dynamic general purpose 
programming language, that uses an interpreter, and which can be used in a vast domain 
of applications\cite{10.5555/2846189}. Internally, Python converts the source code 
into an intermediate form called bytecodes and then translates this into the native 
language of the computer to run. The usual bytecode interpreter of Python 
is CPython\cite{shaw2021cpython}, which is written in C. There are some 
other compilers that are written in python, like PyPy.

Now, there are several python debuggers available, in this section we are going to do a 
small review on some of them, specifically, looking for those that can give us the features 
required to debug \ac{RL} programs, which are, specifically: the ability to navigate through
an execution history (like a time travel debugger), and the ability to modify the values
of variables and resume the execution from a specific point in time, just like 
DeloreanJs\cite{delorean23} debugger.

% PDB
Let us start with the \ac{PDB}\cite{python-pdb}, which is the built-in Python Debugger in CPYthon, it is a 
powerful interactive source code debugger for Python programs. It is implemented as a 
module that can be used in Python scripts, and it provides a command-line interface 
for debugging Python programs. \ac{PDB} allows you to set breakpoints, step through 
code, inspect variables, and evaluate expressions. It also provides a post-mortem 
debugging feature that allows you to debug a program after it has crashed. However, 
\ac{PDB} does not have the ability to navigate through an execution history, modify 
values within a variable, or resume execution from a specific point in time, which 
are the features we are looking for in our solution. Additionally, as \ac{PDB} is the main 
python debugger it has been implemented by several IDEs that provide a missing visual 
interface.

% PuDB
PuDB \cite{pudb}is a full-screen, console-based visual 
debugger for Python. PuDB allows you to set breakpoints, step through code, inspect 
variables, and evaluate expressions. Additionally, it also provides a post-mortem debugging feature 
that allows you to debug a program after it has crashed. However, PuDB does not 
have the ability to navigate through an execution history back in time, modify values 
within a variable, or resume execution from a specific point in time. Nevertheless, this 
can be simulated placing breakpoints in the code, and restarting its execution till the 
breakpoint is reached. The UI makes it possible to go through the code and see the 
variables and the stack trace, and it is very user-friendly.

% PyTrace
PyTrace\cite{pytrace} is a time travel recorder/analyzer for Python. It records code execution, 
variables and stack frames. It has a UI that allows you to navigate through the code, 
see the variables and the stack trace, and it is very user-friendly. However, it is not 
a full-featured debugger, it does not allow you to change the values of variables, or do 
any other operation that a debugger would allow you to do. It is more like a tool to 
understand the behavior of the program, and it allows you to go back and forth in time, 
having different graphs to see the state of your variables at that moment.

% RevPDB
\ac{RevPDB}\cite{revdeb} is a reverse debugger for Python programs, it is made using PyPy as interpreter,
and it allows you to go forward and backward in time, which is an uncommon feature, at least 
in the open source world. However, it is not a full-featured debugger, it is more like a 
proof of concept, and it is not as powerful as \ac{PDB} or PuDB. \ac{RevPDB}, it is not 
currently maintained, its last update was done in 2016, and it does not as 
user-friendly UI as the other debuggers.

% UDB
\ac{UDB}\cite{udb} is a propietary debugger, developed by the Company 
Undo, for Python programs that allows you to go back in time 
and inspect the state of the program at a previous moment. \ac{UDB} works on the principle of first recording 
a program while it is running normally, and then replaying the execution while allowing the 
user to navigate and inspect state at different points in time. They work at the process 
level, rewinding and replaying the state of the entire Linux (or Windows) process. Technically, 
\ac{UDB} does allow new code paths to be executed at replay-time, but these executions are 
isolated.

\fref{tab:python-debuggers} represents a comparison between the different Python debuggers, 
and the features that they offer. 

% Please add the following required packages to your document preamble:
% \usepackage{graphicx}
\begin{table}[H]
\centering
\resizebox{\columnwidth}{!}{%
\begin{tabular}{c|lllll}
\textbf{Features}               & \multicolumn{1}{c}{\textbf{PDB}}                                                         & \multicolumn{1}{c}{\textbf{PuDB}}                                                                               & \multicolumn{1}{c}{\textbf{PyTrace}}                                           & \multicolumn{1}{c}{\textbf{RevPDB}}                                                   & \multicolumn{1}{c}{\textbf{UDB}}                                                                                                          \\ \hline
\textbf{Description}            & \begin{tabular}[c]{@{}l@{}}Terminal debugger based\\ on CPython interpreter\end{tabular} & \begin{tabular}[c]{@{}l@{}}UI Debugger that extends\\ PDB debugger\end{tabular}                                 & Post moretm debugger                                                           & \begin{tabular}[c]{@{}l@{}}Terminal debugger based\\ on PyPy interpreter\end{tabular} & \begin{tabular}[c]{@{}l@{}}Not open source debugger,\\ that creates functionalities \\ for back in time debugging\\ for pdb.\end{tabular} \\
\textbf{Stepping}               & It allows stepping                                                                       & It allows stepping                                                                                              & It allows stepping                                                             & It allows stepping                                                                    & It allows stepping                                                                                                                        \\
\textbf{Breakpoints}            & It allows breakpoints                                                                    & It allows breakpoints                                                                                           & Doesn't allow breakpoints                                                      &                                                                                       & It allows breakpoints                                                                                                                     \\
\textbf{Stepping back}          & \begin{tabular}[c]{@{}l@{}}It can be simulated using \\ jumps through time\end{tabular}  & \begin{tabular}[c]{@{}l@{}}It can be simulated restarting\\ the execution and using \\ breakpoints\end{tabular} & It allows stepping back                                                        & It allows stepping back                                                               & It allows stepping back                                                                                                                   \\
\textbf{Variables modification} & \begin{tabular}[c]{@{}l@{}}It allows variables\\ modification\end{tabular}               & \begin{tabular}[c]{@{}l@{}}It allows variables\\ modification\end{tabular}                                      & \begin{tabular}[c]{@{}l@{}}Doesn't allow variables\\ modification\end{tabular} & \begin{tabular}[c]{@{}l@{}}It allows variables\\ modification\end{tabular}            & \begin{tabular}[c]{@{}l@{}}It allows variables\\ modification\end{tabular}                                                                \\
\textbf{UI}                     & \begin{tabular}[c]{@{}l@{}}In it's simple form does \\ not have a UI\end{tabular}        & Does have a UI                                                                                                  & Does have a UI                                                                 & \begin{tabular}[c]{@{}l@{}}In it's simple form does \\ not have a UI\end{tabular}     & Does have a UI                                                                                                                           
\end{tabular}%
}
\caption{State of the art Python debuggers}
\label{tab:python-debuggers}
\end{table}


\section{State of the art in \ac{AI} programs}
\label{sec:ai}

\ac{ML} programs are often debugged using visualization 
tools that allow developers to observe the program's behavior and analyze its internal state. 
However, most of the time these tools are postmortem and do not allow developers to interact 
with the program during execution, this makes these tools not suitable for \ac{RL} as they 
do not consider that \ac{RL} programs are high dimensional models and are hard to understand, 
due to a stronger feedback loop between the researcher and the agent; different 
supervised learning, which involve a static dataset. Nonetheless, they 
represent an important step in the state-of-the-art for the creation of visualization 
and debugging tools for \ac{RL} programs.

TensorBoard\cite{tensorboard} is a 
visualization tool for TensorFlow, one of the most popular machine learning libraries. 
TensorBoard allows developers to visualize the behavior of their machine learning models, 
including training graphs, variable distributions, and model structure visualizations. However, 
TensorBoard does not allow developers to interact with the model during execution, which is 
essential for debugging \ac{RL}.

Another implementation for debugging \ac{RL} programs is \ac{Vizarel} \cite{deshpande2020interactivevisualizationdebuggingrl}.
\ac{Vizarel} implements an interactive visualization tool for debugging and interpreting \ac{RL} 
programs. The system offers different views that encapsulate the spatial and temporal dimensions 
of agent policies. The tool consists of a set of "viewport" modules, each of which is an abstract 
entity that can be backed by different specs, conditioned on the underlying data\cite{deshpande2020interactivevisualizationdebuggingrl}. This tool 
is useful to understand the behavior of the agent, however, it does not allow the user to interact with the 
variables during execution, which leads to spend time in the training process without knowing if
the agent is learning correctly or not.

\citet{Rajan_2023} presents a tool 
called \ac{MDP} Playground. This platform is intended to help developers and researchers to understand
\ac{RL} agents better on toy complex environments, and to create unit tests characteristics of agents 
on toy examples of \ac{MDP}s. Basically the problem that they identified was that \ac{RL} programs 
are fragile and hard to debug, and with deep \ac{RL} research growing, the complexity of the 
dynamics has increased and researchers have not paid enough attention to the state distribution and 
reward distribution which \ac{RL} policies work and that has made \ac{RL} programs hard to debug 
and fragile\cite{Rajan_2023}. The idea with this is to open the box in terms to understand the behavior of the agent 
while using toy examples, based on the OpenGym library. This work is not scalable, and it is not 
suitable for real-world applications, but it is a good starting point to understand the behavior of 
\ac{RL} agents. While the \ac{MDP} Playground is not a debugger, it can help to debug the program by understanding the 
behavior of the agent in a controlled environment.

Finally, another recent work to debug \ac{RL} programs by \citet{Steinmetz2021DebuggingAP},
which tries to gain trust in \ac{AI} programs using systematic testing. Erroneous program behavior
(called bugs) becomes bad policy behavior, for example, not reaching the 
goal, or not learning the correct policy. They propose a systematic testing approach to find these 
bugs in which they address how to confirm a bug is found and how potential bugs can be found, 
using the Framework Fuzzing\cite{Steinmetz2021DebuggingAP}.

While the approaches presented are very interesting to debug \ac{RL} programs, there is still
a gap in the current literature, which is addressed as the focus of this work. The idea 
of this work is to see how 
a debugger that allows the user to navigate through the execution history, modify the values of 
variables, and resume the execution from a specific point in time, performs into creating
better \ac{RL} programs.

\endinput

