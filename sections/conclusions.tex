% $Id: conclusions.tex 1784 2012-04-27 23:29:31Z nicolas.cardozo $
% !TEX root = main.tex

\chapter{Conclusions and Future Work}
\label{cha:con-fut}

\section{Conclusions}
\label{sec:conclusions}

In this work, we presented \ac{Flik}, a back-in-time debugger for \ac{RL} programs.
We initially had two major problems: First, \ac{RL} programs were very complex in terms of the environment,
so the developer would have major problems in understanding the behavior of the agent, and
the possible bug being presented, in the program. Also the complexity of the program, requires
too much computational power which makes it even harder to interact to. This Lead us to our second problem,
the interaction between the agent and the environment is very hard to understand, and
for a developer that needs to debug the program, it is very hard to understand the behavior of the agent
and the interaction that a possible bug is causing.

A back in time debugger strategy showed being a really good strategy for developing and 
understanding RL programs. Most of the developers managed to understand the behavior of the agent,
and find the bug hidden on it. Specifically, \ac{Flik}, a console-based debugger, demonstrated to 
be a very useful tool to find general bugs that an \ac{RL} program can present during development. 
The tool that allowed developers to interact with the program during 
execution, inspect the internal state of the agent, and modify its behavior in real-time. These were 
important features as we wanted \ac{Flik} to be able to tackle the problem of interaction with the 
program to be able to inspect in a deeper way the state of the program in terms of variables, and 
it provides a functionality of interacting with the execution of the program going forward and backwards.
Additionally, \ac{Flik} had a visual interface, which reduced the complexity of the \ac{RL} program,
making it easier for the user to inspect the state step by step.

\section{Future Work}
\label{sec:future-work}

As future work, on proposal form the student's feedback was to optimize memory consumption by 
using layer cashing in a similar was as git and docker way save states, so instead of saving the
history of the entire variables (globals and locals), and all the metadata of the program, we could 
only save the changes made on the program, this could optimize the memory consumption of \ac{Flik},
nevertheless, this would require a more complex logic to be implemented. 

Moreover, the students suggested that the visual interface could be improved, as it was not very 
intuitive for them to use it (taking into account most of them were not used to use a debugger).
\ac{Flik} could have a VSCode pluging to be added to the editor, so the user could use the debugger 
in a more friendly way, and also have the possibility to use the debugger in a more visual way.

Finally, \ac{Flik} could be improved by integrating the environment visuals made in matplotlib
graphs or pygame graphs in the visual interface, so the user could have a better understanding of 
the environment and the agent's behavior, while debugging the program.

