% $Id: introduction.tex 1784 2012-04-27 23:29:31Z nicolas.cardozo $
% !TEX root = main.tex

\chapter{Solution}
\label{cha:solution}

As mentioned earlier, the black box problem is not new, and it is something 
that can be addressed using a debugger. However, currently, there is no 
debugger capable of working with RL programs. Therefore, the proposed solution 
in this work is the creation of a debugger specifically for RL programs. This 
is not a traditional debugger; rather, it allows for an understanding of the 
internal state of the agent, the decisions it makes, and the rewards it 
receives. In other words, it understands the execution context of the agent 
in terms of variables, environment, and rewards. This enables the developer 
to interact with the program during execution, modify the values of variables, 
and continue the program's execution.

Additionally, this aims to provide a deeper understanding of the behavior of 
RL programs and to identify errors that arise during the learning process. 
This would allow for the evaluation of the construction and quality of 
software developed for RL and help formulate strategies to improve the 
development of these programs.

Thus, this solution proposes a framework that enables developers to 
analyze RL programs, evaluate their behavior, and observe the evolution 
of different variables over time.

\endinput

